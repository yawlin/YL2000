\def\tit#1{
 \begin{center}\Large \bfseries
 #1
 \end{center} \vspace*{.2in}
}

\documentclass[12pt]{article}
\begin{document}

\tit{Problem $L$ --- A Version of Nim Game}

\noindent

Here we introduce a special version of the combinatoric game called {\em
Nim}. This progressively finite take-away game involving 4 piles of
stones. In this game, the players take turns removing at least one but at
most 3 stones from one of the piles. The player who takes the last stone
from the table losses.

The configuration of an instance of the Nim game can be described with
four nonnegative integers representing the sizes of these four piles. That
is, the configuration in a Nim game requires a vector of nonnegative
integers $(p_1, p_2, p_3, p_4)$, the $k$th number $p_k$ representing the
current size of the $k$th pile.

To write a program that plays the Nim game perfectly, we need to decide
whether a given instance of the Nim game is winnable. For example, the
configuration $(0, 0, 0, 2)$ is winnable by removing one stone from the
last pile; however, you can verify that neither $(0, 0, 0, 5)$ nor $(2, 2,
0, 0)$ is winnable. Further, to make the problem easier, we assume that
the number of stones on each pile is at most 9 stones.

\vspace*{.1in} \noindent {\large \bfseries Input File} \vspace*{.1in}

\noindent {\em Several} instances of Nim game configurations. The inputs
are a list of integers. Within each set, the first integer (in a single
line) represents the number of Nim game configurations, $n$, which can be
as large as 20. After $n$, there will be $n$ lists of Nim game
configurations; each line contains four integers $\langle p_1, p_2, p_3,
p_4\rangle$. Note that integers $0\leq p_1, p_2, p_3, p_4\leq 9$.

\vspace*{.1in} \noindent {\large \bfseries Output File} \vspace*{.1in}

\noindent For each configuration appeared in the input, decide whether it
is winnable. In particular, output a single number `{\tt 1}' if the
instance is winnable; otherwise, output a single number `{\tt 0}' if it is
not winnable.

\vspace*{.1in}
\noindent
{\large \bfseries Sample Input}
\begin{verbatim}
 4
 0 0 0 5
 0 0 0 6
 0 0 2 2
 1 2 3 4
\end{verbatim}

\vspace*{.1in}
\noindent
{\large \bfseries Sample Output}
\begin{verbatim}
 0
 1
 0
 0
\end{verbatim}

\end{document}
